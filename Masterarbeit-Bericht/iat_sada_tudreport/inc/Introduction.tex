
\chapter{Introduction}\label{cha:Intro}

\section*{1.1. Introduction}
\label{sec:Introduction}
\addcontentsline{toc}{section}{1.1. Introduction}

Technology is also developing in the automotive industry like in other industries over
time. The number ofsensors and accordingly features in automobiles increases 
exponentially. One of the these sensors is color camera. At the beginning, the cameras
were used in the automobile industry just for helping to park and to drive backwards. 
Nowadays, one of the main functions of color cameras is lane detection in autonomous 
cars and in the cars which have lane departure warning system.
In this master thesis must be the lanes detected and formulated mathematically.

The results of this master thesis will be expanded by the students, which will be
participated to the Echtzeitsysteme Projektseminar at the Technical University of 
Darmstadt. One of the aim of this seminar is to attend the Carolo-Cup which is organized 
annual by Technical University of Braunschweig. Because of that, in this master thesis, 
width of track, its curvature and changes of curvature of which, are used from 
Carolo-Cup. In real life, they are more problems for detecting lanes, they are shadows 
of which can be caused by appearances of trees and buildings, the dirt left on the road 
surface etc.

Thereby, the lanes of track must be detected in enough short time and there shouldn't
be any dead time between lane detection and formulating mathematically. Other main thing,
the lane detection must be enough robust sein. The lanes must be detected also certainly 
with expected light conditions.

\section*{1.2. Problem Statement and Objective Target}
\label{sec:Problem Statement and Objective Target}
\addcontentsline{toc}{section}{1.2. Problem Statement and Objective Target}

Autonomous driving is one of the active researching themes. Autonomous driving can be 
studied in two fundamentel tasks. These tasks are : lane detection and lane guidance. 
For lane detection, they are different scientific techniques in the literature. All of 
these techniques have advantages and disadvantages for different situations. For example,
some of these techniques are suitable just "lines"  not for curves. Some of them are 
suit for also curves but they are not good in different light conditions and some of them
are robust and suit for curves but they are so much computationally expensive. In these
master thesis, our aim is, to research and implement the best method for Carola-Cup.


\section*{1.3. Structure of Thesis}
\label{sec:Structure of Paper}
\addcontentsline{toc}{section}{1.3. Structure of Paper}

In Chapter \ref{cha:Fundamentals} is explained the fundamentals of lane detection. 
All methods, which are used in this thesis, are also explained in this chapter with 
their reasons. Some methods are also compared with their advantages and disadvantages.

In Chapter \ref{cha:Implementation} is explained steps of implementation. These steps
are properties of track, hardware of model car and used software libraries and 
programs. In this chapter will also explained detailed program flow.

In Chapter \ref{cha:Evaulation and Discussion}

In Chapter \ref{cha:Related Works}

In Chapter \ref{cha:Conclusion}
