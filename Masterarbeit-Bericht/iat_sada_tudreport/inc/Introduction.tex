
\chapter{Introduction}\label{cha:Intro}

Automobiles have been an essential part of modern life for the better part of a century and have, for just as long, been a source of injury and death due to accidents. According to data from  Federal Statistical Office in Wiesbaden, in 2016, roughly 2.6 million road traffic accidents occured in Germany and because of these accidents, 3,206 people died. When compared with 2015, although the number of deaths due to traffic accidents decreased by 7.3\%, the number of road traffic accidents increased by 2.7\% in 2016.\cite{Statis}

The number one cause of traffic accidents is human error. Like all industries, the automotive industry continues to change and develop rapidly. In a couple of years, there will be more autonomous cars on the road and these cars will eventually replace human drivers. Because of this reason, a large portion of road traffic accidents will be eliminated. This is not the only advantage of autonomous cars. Thanks to autonomous cars, people in traffic will experience less stress, and will also have more time for other things. While driving, the people will be able to work, eat, read and even sleep. But it is also not so easy to build such reliable cars. Because of this reason, nowadays, one of the biggest research areas in the automotive industry is autonomous cars. This research area includes many different fields. Some of these fields are: Car-2-Car/Car-2-X communication, lane detection, sign recognition, object detection, path planning, and so on. In this master's thesis, some lane detection methods, their implementations, advantages, and disadvantages will be discussed.

As in all industries, technology in the automotive industry is continuing to develop day by day. For example, the number of sensors, and their corresponding features, is increasing exponentially. One such sensor is the color camera. To begin with, in the automotive industry, cameras were used only to assist drivers in parking and reversing.
 
Nowadays, however, one of the main functions of color cameras is lane detection, in both autonomous cars and in cars equipped with a lane departure warning system. In this master's thesis, the lanes will be detected and then formulated mathematically.

The results of this master's thesis will be utilized and expanded upon by the students who will participate in the Echtzeitsysteme Projektseminar at the Technical University of Darmstadt. One of the aims of this seminar is to attend the Carolo-Cup organized annually by the Technical University of Braunschweig. Because of that, the width, the curvature, and the changes of the curvature of the track used in this master's thesis are the same as those belonging to the track used in the Carolo-Cup. In a real-life situation, there are of course oftentimes more factors that can hinder lane detection, including shadows cast by trees, buildings, and other structures; sunlight directly entering the lens of the camera and similarly less-than-ideal lighting conditions; dirt and debris on the road surface; and so on.

Therefore, the lanes of the track must be detected in a sufficiently short amount of time and there should be no dead time between lane detection and mathematical 
formulation. Lane detection must also be sufficiently robust, so that it should not be disrupted by less-than-ideal lighting conditions.


\section{Carolo-Cup}\label{sec:Carolo-Cup}

\emph{\color{red}The Carolo-Cup is a student competition, providing student teams with a platform for the design and implementation of automated RC cars. The main challenge is to implement cutting-edge algorithmic solutions for vehicle control and environment perception, based on a realistic application scenario.}

\emph{\color{red}In the annual competition, the students will present their solutions to a jury from academia and industry, while competing with other international teams from different universities.}

\emph{\color{red}Each student team is put in charge of developing, producing and demonstrating a cost- and energy-efficient 1:10 concept of an automated vehicle by a fictional OEM.}

\emph{\color{red}During the competition several driving tasks have to be executed as fast and precisely as possible.}

\emph{\color{red}In addition, the developed concept must be presented and explained.}

\emph{\color{red}In 2017 additional challenges have been introduced: The teams must not only stop at intersections and evade obstacles on the road, but also recognize and adhere to traffic signs. This enables more complex situations at intersections and shall provide an even more realistic urban setting.}



\section{Problem Statement and Objective Target}\label{sec:Problem Statement and Objective Target}


Autonomous driving is a topic currently being actively researched. Research on autonomous driving can be conducted in two fundamental areas: lane detection and lane guidance. With regard to lane detection, there are different scientific techniques that can be utilized, according to the literature, all with their own advantages and disadvantages under different conditions. For example, some techniques are suitable for straight lines, but not for curves. Others are suitable for curves as well but do not function well under certain light conditions. Others still are quite robust and suitable for curves, yet are computationally intensive (resulting in a video feed with significant gaps). 

In this master's thesis, my aim is to research and implement the most appropriate and effective method for use in the Carola-Cup.


\section{Structure of Thesis}\label{sec:Structure of Thesis}


In Chapter \ref{cha:Fundamentals}, the fundamentals of lane detection are explained. All methods utilized in this thesis, along with their respective justifications, are also explained in this chapter. Some methods are also compared with regard to their advantages and disadvantages.


In Chapter \ref{cha:Implementation}, the steps of implementation are explained. The components can be divided broadly into the properties of the track, the hardware of the model car, and the software libraries and programs to be utilized. In the software section, all cases will be explained in detail.In this chapter, the program flow will also be explained in detail.

In Chapter \ref{cha:Evaulation and Discussion}, the results of the methods utilized will be compared. The computing time of all phases in this thesis will be presented and discussed. Also, all parameters utilized and their effects on this thesis will be also presented and discussed. In this chapter, the researcher will attempt to find an answer to the question, 'How can computing time be reduced?'.

In Chapter \ref{cha:Related Works}, the state of the art will be discussed. The other possible solutions for lane detection will also be explored here and their advantages and disadvantages will be compared.

In Chapter \ref{cha:Conclusion}, all results of this master thesis will be presented and possible improvements and/or enhancements will be discussed.


