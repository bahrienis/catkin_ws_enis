%
\chapter{Conclusion}\label{cha:Conclusion}
%

The aim of this master's thesis is to implement a method, which detect the lanes specifically for indoor scenarios, which is necessary for the Carolo-Cup. On the other hand, the compution time shouldn't be so high and the project has to work stabil in the expected light conditions. 

To reach the aim, so much projects about lane detection were researched and their advantages and disadvantages were compared with each other. In the literatur, there are so many different method for lane detection and all of them have advantages and disadvantages  	dependent on the application scenario and other factors(detecting just straight lanes or straight and curve lanes, power of the CPU, needed FPS value, and so on). In this case, 5 different methods were implemented. In  3 of the 5 methods, Inverse Perspective Mapping algorithm is used which diminish the view effect but conversely, the computing time of the IPM algorithm is so high. 2 of the 5 methods show that, the lane detection algorithm can be also implemented without IPM algorithm. In all methods, preprocessing parts (Edge detection, threshold filter and so on) are similar. In all methods, for the detecting the pixels which are on the lanes, the Hough Transformation was used. End of the Hough Transformation implementation, the pixels which are close to each other have to been found and for this task, two different algorithms were used which are called K-Nearest Neighbors algorithm and Rectangle algorithm. After all pixels on the lanes are grouped for each lanes, the curve fitting algorithm is used for all lanes in all methods.

After the measurement of the computing times for all algorithms is noticed, that the preprocessing part, the IPM algorithm and Probabilistic Hough Transformation are the most computationally expensive algorithms. In some methods, computing time of these algorithms were descreased. In some methods, the IPM algorithm is implemented just for the fitted curves, in the other words, the IPM algorithm in Method 1 and Method 4 is implememented just for the approximatly 0.5\% of the pixels in the frame so it saves so much computing time for the IPM algorithm.  Another case with the computing time is to use two different algorithms for the grouping the Hough points after the Probabilistic Hough Transformation was used. K-Nearest Neighbors Algorithm takes less computing time to compare to Rectangle Method. The last thing, which can be concluded about computing time is the resolution of the frames. If the resolution of the frames is discreased, the number of the pixels which have to be worked on is also discreased so, the processes take less time.