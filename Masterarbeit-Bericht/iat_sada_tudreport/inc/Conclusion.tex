%
\chapter{Conclusion}\label{cha:Conclusion}
%

The aim of this master's thesis is to implement a method which can detect lanes in an indoor setting, which is necessary for the Carolo-Cup. In addition, the computing time required for lane detection should not be very high, and lane detection must remain stable in the expected light conditions.

In order to achieve this objective, many projects on lane detection were researched and their advantages and disadvantages were compared. In the literature, there are many different methods for lane detection and all of them have advantages and disadvantages, which are dependent on the application scenario as well as other factors (detecting only straight lanes or straight and curved lanes, the power of the CPU, the required FPS value, and so on). In this case, five different methods are implemented. In 3 of the 5 methods, the Inverse Perspective Mapping (IPM) algorithm is used. The IPM algorithm diminishes the view effect but also has a high computing time. Two of the 5 methods show that the lane detection algorithm can be also implemented without the IPM algorithm. In all methods, the preprocessing parts (edge detection, threshold filter, and so on) are similar. In all methods, in order to detect the pixels on the lanes, the Hough Transformation is used. At the end of the Hough Transformation implementation, the pixels which are close to each other have to be found and for this task, two different algorithms are used: the K-Nearest Neighbors algorithm and the Rectangle algorithm. After all pixels are grouped according to each lane on which they lie, the curve fitting algorithm is used for all lanes in all methods.

After measuring the computing times for all algorithms, it is observed that for the preprocessing part, the IPM algorithm and Probabilistic Hough Transformation are the most computationally expensive algorithms. In some methods, the computing times of these algorithms are decreased. In Methods 1 and 4, the IPM algorithm is implemented only with regard to the fitted curves, and thus only for approximately 0.5\% of the pixels in the frame, saving a lot of computing time for the IPM algorithm as a result. It is also possible to compare the computing times with regard to grouping the Hough Points after the Probabilistic Hough Transformation is used. Two different algorithms can be used to group the Hough Points: the K-Nearest Neighbors Algorithm and the Rectangle Method. The K-Nearest Neighbors Algorithm requires less computing time. The last factor with regard to computing time to be discussed is the resolution of the frames. If the resolution of the frames is decreased, the number of the pixels which have to be worked on also decreases, so the processes take less time.

In conclusion, as an outcome of this work, three algorithmic approaches to road lane detection are constructed and tested in a real environment. All of them fit into desired timing requirements and show stable and precise detection results in various light conditions, with Method 3 being the most successful in terms of both quality and performance. Proposed algorithms can be used as a base for autonomous car operation during the Carolo-Cup.