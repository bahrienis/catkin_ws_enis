
%
\chapter{Evaluation and Discussion}
\label{cha:Evaluation and Discussion}

In this chapter, the algorithms for each method used in this master's thesis will be evaluated.
In addition, the parameters and their effects on computing time and the reliability of lane
detection will be observed. The methods will be also compared with each other, with the aim
of finding the most efficient and reliable method. In another section of this chapter, the average
computation times for each of the algorithms as well as for the total lane detection process in
each method will also be measured. At the end of this chapter, the problems which can occur
during lane detection will also be defined.

%

\section{Average Computing Time}\label{sec:Average Computing Time}

In total, five different methods for lane detection were implemented. Each method has a different average computing time as well as a different level of reliability in terms of lane detection. In this chapter, all of the methods, including their respective algorithms, will be described. The computing times of said algorithms, as well as how changing the parameters of the algorithms affects their computing times, will also be discussed.


In the following table, the computing times for all algorithms, as well as their average computing times, are displayed. Due to limited space, abbreviations were used in place of the following terms:


\begin{itemize}[noitemsep]
\item\textbf{PRE : }Preprocessing
\item\textbf{SHT : }Standard Hough Transformation
\item\textbf{FLL : }Finding the Locations of the Lanes
\item\textbf{PHT : }Propabilistic Hough Transformation
\item\textbf{REC : }Rectangle Algorithm
\item\textbf{CF : }Curve Fitting and changed perspective of the fitted curves from the camera side to the top side 
\item\textbf{PUB : }Publishing the coefficients of the fitted curves
\item\textbf{ACT : }Average Computing Time
\end{itemize}


\begin{table}[ht]
\caption{Computing Times of Method 1} 
\centering 
  \begin{tabular}{ | c | c | c | c | c | c | c | c |}
    \hline
  
  PRE & SHT & FLL & PHT & REC & CF & PUB & ACT \\ \hline  
  20.15 ms & 0.81 ms & 2 $\mu$s & 23.42 ms & 0.83 ms & 4.54 ms & 19 $\mu$s& 49.77 ms \\ \hline  
    
    
      \end{tabular}
      \label{tab:Case1_Times}
      \end{table}


As also shown in Table \ref{tab:Case1_Times}, the longest computing times belong to 'Preprocessing phase' and 'Probabilistic Hough Transformation'. These values can be decreased by changing the parameters of Probabilistic Hough Transformation or by removing some steps in the 'Preprocessing phase'. In both cases, however, doing so can also result in undesirable effects. For example, in this project, there is a dynamic thresholding filter utilized during the 'Preprocessing phase'. A function finds the lightest pixel in the the frame and then the threshold value of this filter is changed dynamically according the value of the lightest pixel. Using such a filter results in a faster computing time, but it would also result in less reliability in changing light conditions. With regard to changing the parameters of the Probabilistic Hough Transformation, though it is possible to reduce its computing time, fewer Hough Points on the lanes will be detected. If too few Hough Points are detected, then the stability of lane detection will also suffer.

The advantage of this method is, it is the fastest method, which gets the frames with 640x480 pixels resolution, but the disadvantage of this method is, it is hard to detect the lanes, which are far away from the camera. Because the lanes, which are far away from the camera, look smaller compare to the lanes which are close to the camera.




\textbf{Method 1(b) : }The resolution of the frame from the camera is decreased in this method. The
process in this method is nearly identical to the process in Method 1, but the resolution of the frame is four times smaller. Due to the presence of fewer pixels in the frames, the computing time is much less when compared to Method 1.



\begin{table}[ht]
\caption{Computing Times of Method 1(b)} 
\centering 
  \begin{tabular}{ | c | c | c | c | c | c | c | c |}
    \hline
  
  PRE & SHT & FLL & PHT & REC & CF & PUB & ACT \\ \hline  
  5.26 ms   &  0.34 ms  &  1 $\mu$s  &  11.60 ms  &  0.28 ms  &  3.49 ms &  17 $\mu$s  &  20.99 ms \\ \hline  
    
    
      \end{tabular}
  \label{tab:Case4_Times}
\end{table}

The computing time is the biggest advantage of this method, but there are also some disadvantages of decreasing the resolution of the frames. The size of the lanes is also smaller in this method when compared to Method 1, so the reliability of the lane detection is a bit worse and is also more sensitive to the changing light conditions.

This method should be used if a higher FPS value of the camera is desired. Because of the lower computing time, it is possible to raise the FPS value. This method is also useful if the processor is not powerful enough for a high computing effort.







\textbf{Method 2 : }In the Method 1, the Inverse Perspective Method was used only for the fitted curves.
But in this method, the IPM algorithm was used at the beginning. In the following table, the computing times for each algorithm, as well as the average computing time for this method, are shown. Terms used in Method 2, which were not used in Method 1, and their respective abbreviations are as follows:


\begin{itemize}[noitemsep]
\item\textbf{IPM : }Inverse Perspective Mapping Algorithm
\item\textbf{KNN : }K-Nearest Neighbors Algorithm
\end{itemize}


\begin{table}[ht]
\caption{Computing Times of Method 2} 
\centering 
  \begin{tabular}{ | c | c | c | c | c | c | c | c | c | c |}
    \hline
  
  IPM 		& PRE 		& SHT	   & FLL 	   & PHT 	   & FSP 	    & KNN 	   & CF 	  & PUB 	& ACT \\ \hline  
  15.73 ms & 15.85 ms & 1.16 ms & 5 $\mu$s & 17.51 ms & 11 $\mu$s & 3.02 ms & 3.96 ms & 20 $\mu$s       & 57.27 ms   \\ \hline  
    
      \end{tabular}
  \label{tab:Case2_Times}
\end{table}

As shown in the table above, the average computing time in Method 2 is much higher than in Method 1. One of the biggest reasons for the higher computing time is IPM. In Method 1, the IPM algorithm was used only for the pixels relevant to the fitted curves, but in Method 2, the IPM algorithm was used for the entire frame. In other words, in Method 2, the IPM algorithm is used for many more pixels than in Method 1, resulting in a higher computing time. Another difference between Method 1 and Method 2 is the use of the KNN and the rectangle algorithms, respectively. The KNN algorithm used in the Method 2 is much faster than Rectangle algorithm used in the Method 1.

The main advantage of this method is that, thanks to the IPM algorithm, there is no perspective
effect. In the other words, the size of the lanes does not appear smaller even if they are further away from the camera. The main disadvantage of this method is the computing time. Because of the IPM algorithm, the computing time is higher than in Method 1.



\textbf{Method 2(b) : }This is also a method in which the resolution of the frames is decreased. In this method, the input resolution of the frame is 320x240 pixels directly at the beginning resulting in a resolution that is four times less when compared with Method 2. As a result, the computing time is also less than in Method 2.


\begin{table}[ht]
\caption{Computing Times of Method 2(b)} 
\centering 
  \begin{tabular}{ | c | c | c | c | c | c | c | c | c | c |}
    \hline
  
  IPM 		& PRE 		& SHT	   & FLL 	   & PHT 	   & FSP 	    & KNN 	   & CF 	  & PUB 	& ACT \\ \hline  
  4.61 ms   &  3.66 ms  &  0.21 ms  &  2 $\mu$s  &  5.43 ms  &  6 $\mu$s &  1.95 ms  &  2.09 ms & 17 $\mu$s & 17.99 ms\\ \hline  
    
    
      \end{tabular}
  \label{tab:Case5_Times}
\end{table}


The advantages and disadvantages of descreasing the frame resolution are already mentioned in the Method 1.b description.




\textbf{Method 3 : }This method is identical to Method 2, except for one difference. Instead of the
K-Nearest Neighbors Algorithm, the Rectangle Algorithm is used. As a result, the terms and their respective abbreviations used in the following table are also identical to those used in Method 2, with the exception of REC (previously defined in Method 1), which replaces KNN:


\begin{table}[ht]
\caption{Computing Times of Method 3} 
\centering 
  \begin{tabular}{ | c | c | c | c | c | c | c | c | c | c |}
    \hline
  
  IPM 		& PRE 		& SHT	   & FLL 	   & PHT 	   & FSP 	    & REC 	   & CF 	  & PUB 	& ACT \\ \hline  
  19.36 ms & 18.57 ms & 1.35 ms & 5 $\mu$s & 17.33 ms & 17 $\mu$s & 3.50 ms & 5.12 ms & 20 $\mu$s       & 65.27 ms   \\ \hline  
    
    
      \end{tabular}
  \label{tab:Case3_Times}
\end{table}

With this method, the KNN algorithm and Rectangle algorithm can be compared to each other, because with the exception of when the algorithms are utilized in their respective methods, the results are quite similar. The results show that the KNN algorithm is faster when compared to the Rectangle algorithm. The main advantage of this method is again due to the IPM algorithm, but its high computing time is also the biggest disadvantage of this method.





\section{Lane Detection Quality}\label{sec:Lane Detection Quality}

High computational performance is essential for lane detection, however this parameter alone is not sufficient for the algorithm to be considered successful. Another important criterion is the quality of lane detection. Each of the proposed methods has some advantages and disadvantages from this perspective. For instance, Inverse Perspective Mapping (IPM) algorithm adds more stability to the methods where it is used. That happens because IPM algorithm shows the lane from the bird's eye perspective, so the sizes of the lanes are the same. In Method 1, however, the IPM algorithm was not used, so the lanes' geometry changes depending on the proximity to the camera. That is why when the rectangle method is applied, the size of the rectangles should be adjusted according to the y-coordinate, otherwise the Probabilistic Hough Points from the different lines could be mixed up with each other. Using IPM in Method 3 allows such adjustments to be omitted.

In addition, there are two different algorithms, which are used to group the Probabilistic Hough Points: k-Nearest Neighbors algorithm and the rectangle method. As mentioned in the previous section, KNN algorithm is faster compared to the rectangle method, however in many cases the direction of the curve must be detected to ensure that while choosing the nearest neighbors for the middle line, the Probabilistic Hough Points from the left or the right line are not taken into consideration. Rectangle method is immune to this problem, because the rectangles adjust their coordinates automatically based on the direction of the curves.

Based on these results, the best quality of the lane detection is  achieved by combining IPM algorithm and the rectangle method, which is implemented in Method 3.

