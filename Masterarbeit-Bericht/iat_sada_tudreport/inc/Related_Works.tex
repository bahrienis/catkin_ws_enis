%
\chapter{Related Works}\label{cha:Related Works}

In the automotive industry, one area of particular research interest is autonomous driving, and within this area, lane detection an important component being actively researched. As can be attested by the various projects carried out and scientific papers written on the subject, many different methods can be used in lane detection, each with its own advantages and disadvantages. In terms of disadvantages, some are only able to detect straight lines, for example, while others are not robust in rainy weather or in different light conditions. Some papers also compare the results of different lane detection projects. One of them is from Ammu M Kumar and Philomina Simon from India, which compares some lane detection projects conducted between 2003 and 2014.\cite{Review_of_Lane_Detection}
 
According to the literature, one of the methods used most often for lane detection is the 'Hough Transformation'. But there are also some other methods like 'Haar like features', 'Random sample consensus (RANSAC)' or 'Artificial Neural Network(ANN)'. In this chapter, some different methods will be described in order to provide an overview of the methods used in lane detection.

One important algorithm for lane detection was developed by Mohamed Aly, and was also used by Nicolas Acero Sepulveda, who did his bachelor's thesis in the area of lane detection at the Technical University of Darmstadt in 2016.\cite{Bachelorthesis_Nicolas} In the beginning of this method, with the aid of Inverse Perspective Mapping(IPM) algorithm, the view is changed from the camera view to the top view. By doing so, the perspective effect is avoided. For the top view frame, the Gaussian and threshold filters are used to filter noise. This method is not very different from Method 2 used in this master's thesis. However, in Aly's algorithm the Random sample consensus (RANSAC) algorithm was used for lane detection, while in this master's thesis, the Probabilist Hough Transformation was used.

In another method\cite{An_Efficient_Lane_Detection} for lane detection, the 'Haar like features' algorithm was used to obtain potential lane points. According to authors, the processing time of this method is approximately 0.12 ms and the detection rate is 90.16\%. It is one of the fastest algorithms for lane detection. In this method, because of the perspective effect, the lanes appear diagonal, so a diagonal directional filter called the 'steerable filter' is used, and then the 'Haar like features' algorithm is applied. After this process, the maximum responses are obtained, which are the left and right lanes.

An another robust method\cite{A_Fast_and_Robust_Approach}, which was developed at the Technical University of Braunschweig, was tested successfully in a competition for autonomous automobiles in 2007 called the 'DARPA Urban Challenge'. The 'DARPA Urban Challenge' is different than Carolo-Cup, because the Carolo-Cup is for a 1:10 concept of an automated vehicle, but the DARPA Urban Challenge is for real vehicles. In this project, the IPM technique was used and instead of the RGB color space, the HSV color space was used, because this format is much more robust under different light conditions. In this project, like in Aly's project, the RANSAC algorithm was used. According the paper written for this project, just ten frames per second can be processed and this project needs to have a modern graphics card.

