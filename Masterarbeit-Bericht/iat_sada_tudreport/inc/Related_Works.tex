%
\chapter{Related Works}\label{cha:Related Works}

\emph{\color{blue}Autonomous driving and accordingly lane detection is so active researching area in automotive industry. So in this topic, so many projects were done and about that so many papers were written. This project can be done with so many methods. These methods have some advantages and disadvantages. For example, some of them can detect just the straight lanes, some of them are not rubust in rainy wheather or in different light conditions. There are also some papers, which compare the results of lane detection projects. One of them is from Ammu M Kumar and Philomina Simon from India, which compare some lane detection projects from 2003 until 2014. }\cite{Review_of_Lane_Detection}
 
\emph{\color{blue}Papers show that, one of the most used methods for lane detection is 'Hough Transformation'. But there are also some other methods like 'Haar like features', or Random sample consensus(RANSAC) or Artificial Neural Network(ANN). In this chapter, some different methods will be described and it can give some overview about other implementation methods.} 

\emph{\color{blue}An algorithm for lane detection from Mohamed Aly, which was also used by Nicolas Acero Sepulveda, who did his bachelor's thesis about lane detection at Technical University of Darmstadt in 2016.\cite{Bachelorthesis_Nicolas} In this method, at the beginning with the Inverse Perspective Mapping(IPM) algorithm, the view was changed from camera view to top view. With it, the perspective effect is avoided. For the top viewed frame, Gaussian and threshold filters are used for filter noise. Until now, this method is not so big different than then Method 2 in this master's thesis. In this master's thesis, for detecting the lanes, Probabilistic Hough Transformation was used but in the algorithm from Aly, RRandom sample consensus(RANSAC) algorithm was used for detecting the lanes.}

\emph{\color{blue}In another method\cite{An_Efficient_Lane_Detection} for lane detection, the Haar like features were used to obtain candidate lane points. According to autors, the processing time of this method is approximately 0.12 ms. and the detection rate is 90.16\%. It is one of the fastest algorithm for lane detection. In this method, because of the perspective effect, the lanes appear diagonal so a diagonally directional filter is used, which is called 'steerable filter', and then 'Haar like features'is implemented. After this process, the maximum responses are obtained, which mean, that they are left and right lanes.} 