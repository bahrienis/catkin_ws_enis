%
\chapter{Related Works}\label{cha:Related Works}

\emph{\color{blue}Autonomous driving and accordingly lane detection is so active researching area in automotive industry. So in this topic, so many projects were done and about that so many papers were written. This project can be done with so many methods. These methods have some advantages and disadvantages. For example, some of them can detect just the straight lanes, some of them are not rubust in rainy wheather or in different light conditions. There are also some papers, which compare the results of lane detection projects. One of them is from Ammu M Kumar and Philomina Simon from India, which compare some lane detection projects from 2003 until 2014. }\cite{Review_of_Lane_Detection}
 
\emph{\color{blue}Papers show that, one of the most used methods for lane detection is 'Hough Transformation'. But there are also some other methods like 'Haar like features', or Random sample consensus(RANSAC) or Artificial Neural Network(ANN). In this chapter, some different methods will be described and it can give some overview about other implementation methods.} 

\emph{\color{blue} An algorithm for lane detection from Mohamed Aly which was also used by Nicolas Acero Sepulveda, who di his bachelor's thesis about lane detection at Technical University of Darmstadt in 2016.\cite{Bachelorthesis_Nicolas}} 

