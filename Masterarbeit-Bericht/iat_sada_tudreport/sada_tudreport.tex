% =================================================================================
% Hier ausw�hlen, ob TUD-Design oder nicht
% =================================================================================
\newif\ifTUDdesign
\TUDdesigntrue					% TUD-Design
%\TUDdesignfalse				% F�r Rechner ohne installierte TUDdesign-Pakete
% =================================================================================


% =================================================================================
% Hier Daten f�r studentische Arbeit eingeben
% =================================================================================
\newcommand{\SADATyp}{Master Thesis}
\newcommand{\SADATitel}{Kamerabasierte Fahrbahnerkennung zur automatisierten Fahrbahnf�hrung eines Modellauto}
\newcommand{\SADAStadt}{Darmstadt}
\newcommand{\SADAAutor}{Bahri Enis Demirtel}
\newcommand{\SADABetreuer}{Dr. -Ing. Eric Lenz}
\newcommand{\SADABetreuerII}{}
\newcommand{\SADABetreuerIII}{}
\newcommand{\SADABegin}{02. May 2017}
\newcommand{\SADAAbgabe}{02. November 2017}
\newcommand{\SADASeminar}{15. November 2017}
% =================================================================================


% =================================================================================
% Auswahl des IAT-Fachgebiets (rtm / rtp)
% =================================================================================
\newif\ifrtm
\rtmtrue	% rtm
%\rtmfalse	% rtp
% =================================================================================


% =================================================================================
% Erkl�rung, dass die Arbeit ohne Hilfe Dritter etc. erstellt wurde
% =================================================================================
\def\SADAVarianteErklaerung{ETIT}		% FB 18, Elektrotechnik
%\def\SADAVarianteErklaerung{MBDA}		% FB 16, Maschinenbau, Diplomarbeit
%\def\SADAVarianteErklaerung{MBSA}		% FB 16, Maschinenbau, Studienarbeit
% =================================================================================


% =================================================================================
% Ausnahmen von der automatischen Silbentrennung
% =================================================================================
\hyphenation{Aktu-ali-sie-rung Screen-shots Pa-rallel-ro-bo-ter Zu-stands-raum-mo-del-le nach-voll-zieh-bar Pro-jekt-se-mi-nar}
% =================================================================================


% =================================================================================
% Hier NICHTS �ndern!
% =================================================================================
\ifTUDdesign
	% Ggf. Option "numbers=noenddot" einf�gen, damit "Abbildung 3.1: _" statt "Abbildung 3.1.: _" verwendet wird.
	\documentclass[11pt, twoside, colorback, accentcolor=tud2c, nopartpage, bigchapter, fleqn, english, longdoc]{tudreport}
\else
	\documentclass[11pt, a4paper, twoside, fleqn, ngerman]{scrreprt}
  % F�r Entwurf auf Rechnern ohne installierte TUDdesign-Pakete	
	\usepackage{exscale}	% Korrektur math-Zeichen
	\usepackage{eurosym}

\fi
\input{common/includes.tex}				% verwendete Pakete einbinden
\input{common/setup.tex}					% LaTeX-Einstellungen
\input{common/commonmacros.tex}		% oft verwendete Befehle
% =================================================================================


% =================================================================================
% Hier beginnt das eigentliche Dokument
% =================================================================================
\begin{document}
\input{common/preface.tex} % Titelseite, Aufgabenstellung, Erkl�rung, Abstract, Inhaltsverzeichnis, etc.


\chapter{Introduction}\label{cha:Intro}

\section{Introduction}\label{sec:Introduction}

\emph{\color{blue}Autos are taking so important place in our lives. But every year, so many traffic accidents occur and because of this reason so many people die. According the datas from  Federal Statistical Office, Wiesbaden, in 2016, roughly 2.6 million road traffic accidents occured in Germany and because of these accidents, 3,206 people died. Compare to 2015, the road traffic accidents increase 2.7\% in 2016 but on the other hand, the number of dead people because of traffic acciendt decrease 7.3\% in 2016 compare to 2015\cite{Statis}.}

\emph{\color{blue}The biggest reason of these accidents are caused by humans. Like in all areas, automotive industry changes also so fast. After a couple of years, there will be more autonomous cars at the roads and these cars will take human's place at steering the cars. Because of this reason, a major part of road traffic accidents will be arrested. It is not the only advantage of autonomous cars. Thanks of autonomous cars, the people on the traffic will have less stress, and more time for other things. While driving, the people will can work, eat, read and even sleep. But it is also not so easy to build such reliable cars. Because of this reason, nowadays, one of the biggest research area of automotive industry is autonomous cars. This research area includes so many different fields. Some of these fields are : Car-2-Car/Car-2-X communication, lane detection, sign recognition, object detection, path planning, etc. In this master thesis, some lane detection methods, their implementations, advantages and disadvantages will be discussed.}

As in all industries, technology in the automotive industry is continuing to develop day by day. For example, the number of sensors, and their corresponding features, is increasing exponentially. One such sensor is the color camera. To begin with, in the automotive industry, cameras were used only to assist drivers in parking and reversing.
 
Nowadays, however, one of the main functions of color cameras is lane detection, in both autonomous cars and in cars equipped with a lane departure warning system. In this master's thesis, the lanes will be detected and then formulated mathematically.

The results of this master's thesis will be utilized and expanded upon by the students who will participate in the Echtzeitsysteme Projektseminar at the Technical University of Darmstadt. One of the aims of this seminar is to attend the Carolo-Cup organized annually by the Technical University of Braunschweig. Because of that, the width, the curvature, and the changes of the curvature of the track used in this master's thesis are the same as those belonging to the track used in the Carolo-Cup. In a real-life situation, there are of course oftentimes more factors that can hinder lane detection, including shadows cast by trees, buildings, and other structures; sunlight directly entering the lens of the camera and similarly less-than-ideal lighting conditions; dirt and debris on the road surface; and so on.

Therefore, the lanes of the track must be detected in a sufficiently short amount of time and there should be no dead time between lane detection and mathematical 
formulation. Lane detection must also be sufficiently robust, so that it should not be disrupted by less-than-ideal lighting conditions.

\section{Problem Statement and Objective Target}\label{sec:Problem Statement and Objective Target}


Autonomous driving is a topic currently being actively researched. Research on autonomous driving can be conducted in two fundamental areas: lane detection and lane guidance. With regard to lane detection, there are different scientific techniques that can be utilized, according to the literature, all with their own advantages and disadvantages under different conditions. For example, some techniques are suitable for straight lines, but not for curves. Others are suitable for curves as well but do not function well under certain light conditions. Others still are quite robust and suitable for curves, yet are computationally intensive (resulting in a video feed with significant gaps). In this master's thesis, my aim is to research and implement the most appropriate and effective method for use in the Carola-Cup.


\section{Structure of Thesis}\label{sec:Structure of Thesis}


In Chapter \ref{cha:Fundamentals}, the fundamentals of lane detection are explained. All methods utilized in this thesis, along with their respective justifications, are also explained in this chapter. Some methods are also compared with regard to their advantages and disadvantages.


In Chapter \ref{cha:Implementation}, the steps of implementation are explained. The components can be divided broadly into the properties of the track, the hardware of the model car, and the software libraries and programs to be utilized. \emph{\color{blue}At software part, the all cases will be in detail explained.}In this chapter, the program flow will also be explained in detail.

In Chapter \ref{cha:Evaulation and Discussion}, the results of the methods utilized will be compared. The computing time of all phases in this thesis will be presented and discussed. Also, all parameters utilized and their effects on this thesis will be also presented and discussed. In this chapter, the researcher will attempt to find an answer to the question, 'How can computing time be reduced?'.

In Chapter \ref{cha:Related Works}, the state of the art will be discussed. The other possible solutions for lane detection will also be explored here and their advantages and disadvantages will be compared.

In Chapter \ref{cha:Conclusion}, all results of this master thesis will be presented and possible improvements and/or enhancements will be discussed.



%
\chapter{Fundamentals}\label{cha:Fundamentals}
%
%Eine wissenschaftliche Abschlussarbeit kann im Allgemeinen in die folgenden 4 Phasen gegliedert werden.
%
\section{Properties of Truck at Carola-Cup}\label{sec:Properties of Truck at Carolo-Cup}
%
The Carolo-Cup is an annual competition at the Technical University of Braunschweig which are attended by students. Every year the truck and some properties of the competition are changing. For example, in the competitions until 2017 there was no traffic sign, but starting in 2017 there are also some traffic signs, speed limit zones, blocked areas and crosswalks for pedestrian. Because of this, in the competitions until 2017, there was only one way to understand who had the right of way. If there is a stop line on the road in front of an intersection, it means the car has to wait until the intersection is free. In the competitions starting from 2017, the intersections are in different parts: They are 'Intersections with stop lines', 'Intersections with give-way lines', 'Intersections with priority to right', 'Enforced crossing direction - give-way condition', 'Enforced crossing direction - right of way condition'. Except 'Intersections with priority to right', they all have traffic signs stating who has priority. If there is a no traffic sign, it means the right side always has priority.\cite{Carolo_Cup}

\begin{figure}[H]
	\centering
	\hspace*{0cm}   
	\includegraphics[width=120mm,scale=1]{./Bilder/Intersections.png}
	\caption{Left: Markings for sharp turns at Carolo-Cup.
Right: Intersections with stop lines at Carolo-Cup\cite{Carolo_Cup}}
\end{figure}

%
\section{Inverse Perspective Mapping}\label{sec:Inverse Perspective Mapping}
%
Inverse Perspective Mapping(IPM) is an algorithm which is able to obtain accurate bird's-eye view images from the sequential of forward looking cameras. With the IPM algorithm, each image pixel is remapped, and a new array of pixels is created where the lines in perspective are transformed into straight lines and objects are distorted. IPM is one of the most used methods in lane detection. In lane detection, IPM ensures that the lanes are shown vertical and parallel to each other. On the other hand, because of the re-mapping of pixels, IPM is a computationally expensive method. Because of this reason, in some cases in this master's thesis, rather than remapping all pixels of the images, only the pixels relevant to the lane and accordingly, the fitted curve, were remapped. Thanks to this, in some cases, a lot of computing time was saved.

In order to use the IPM method, the intrinsic and extrinsic parameters of camera are necessary to process images for coordinate transformation and calibration.

\begin{itemize}

\item \textbf{Intrinsic Parameters :} Intrinsic parameters are camera-specific. It includes
information of the focal length ($f_x$, $f_y$) and optical centers ($c_x$, $c_y$). It is also called a camera matrix. Although the intrinsic parameters are camera-specific, once the camera is calibrated, the modified intrinsic parameters can be stored for future purposes. It is expressed as a 3x3 matrix:

 \begin{center}
  camera matrix =  $
 \begin{bmatrix} 
f_x & 0 & c_x \\
0 & f_y & c_y \\
0 & 0 & 1 \\
\end{bmatrix}
$  \end{center}

\item \textbf{Extrinsic Parameters :} Extrinsic parameters are dependent on the camera position. The parameters are H and $\theta$. H is the distance between the camera and ground. $\theta$ is the camera tilt angle. 
 
\end{itemize}
  
\begin{figure}[H]
\centering
  \includegraphics[width=0.8\textwidth]{./Bilder/Related_positions_of_the_camera.png}\label{Procedures_of_IPM}
  \caption{Related Positions of the Camera \cite{IPM}}
\end{figure}

 As seen at \ref{Procedures_of_IPM}, the camera on the car has field of view 2(FOV2) at the real position of the camera but in this case, the view is not a bird's-eye view, so if the same FOV is to be observed from a bird's-eye view, IPM will virtually change the position to Virtual Position 2 of the camera. In this case, although the camera is at its real position, it will appear as though it is at Virtual Position 2. For that, the image coordinates must also be changed. Below, the steps of IPM calculations from the paper of \cite{IPM} will be detailed.
 
In the formula, the original image coordinates will be defined as (x,y), the destination image coordinates will be defined as ($x^*$,$y^*$), the distance between the ground and the camera will be defined as H, the focal length of camera will be defined as f, and the tilt angle of camera will be defined as $\theta$.
 
\begin{center}
 $x^* = H \frac{x sin \theta + f cos \theta}{-y cos \theta + f sin \theta}$ ;
 $y^* = H \frac{y sin \theta + f cos \theta}{-y cos \theta + f sin \theta}$ 
\end{center}

In this equation, the transformed component values of $x^*$ and $y^*$ may be less than or equal to zero. Because of 
this reason, a constant d is defined as $
\begin{vmatrix}
H(sin \theta + cos \theta)/(f sin \theta - cos \theta) 
\end{vmatrix}
$  + 1. This means that the coordinate point in the original source image has been mapped into the point of the destination image coordinate system. Below there is the proposed equation :
 
 \begin{center}
 $x^* = H \frac{x sin \theta + f cos \theta}{-y cos \theta + f sin \theta}$ + d ,
 $y^* = H \frac{y sin \theta + f cos \theta}{-y cos \theta + f sin \theta}$ + d ,
 where d = 
 $\begin{vmatrix}
 \frac{H(sin \theta + f cos \theta)}{f sin \theta - cos \theta}
 \end{vmatrix}$ + 1
\end{center}

\begin{figure}[H]
 \centering
  \includegraphics[width=1\textwidth]{./Bilder/Procedures_of_IPM.png}\label{Procedures_of_IPM}
  \caption{Procedures of IPM}
\end{figure}


%
\section{Edge Detection}\label{sec:Edge Detection}
%

Edge detectors are essential parts of most computer vision systems. Edge detectors dramatically decrease the amount of data to be processed and extract the useful parts of images. They work by detecting discontinuities in brightness. In this project, the edge detector was used in order to detect the lanes and to exclude unnecessary information from images. There are different methods for edge detection, but they can be grouped into two categories. They are :

\begin{itemize}

\item \textbf{Gradient method : } This method searches for the maximum and minimum in the first derivative of the image and with that, the edges can be found. For this method, the first order derivative filter must be used. For example : Sobel-Operator.
 
\item \textbf{Laplacian method : } This method searches for the zero crossing in the second derivative of the image and with that, the edges can be found. For this method, the second order derivative filter must be used. For example : Laplacian Filter. 
  
\end{itemize}
 
According to \cite{Machine_Vision},here are three steps of the edge detection algorithm. They are :

\begin{itemize}

\item \textbf{Filtering : } For edge detection, it is required to use a suitable smoothing filter. The filters sharpen the edges and ignore the unnecessary information. It is often utilized to improve the functioning of an edge detector against noise. The more filtering is applied, however, the greater the loss of edge strength.
 
\item \textbf{Enhancement : } To be able to better detect edges, changes in the intensity in the area surrounding a point must be determined. Pixels in which a significant change in intensity occurs are emphasized by enhancement, which is usually applied by calculating the gradient magnitude.
  
\item \textbf{Detection : } Though many points in an image have a nonzero value for the gradient, not all of these points are actually edges. Because only points with strong edge content are desired, a method must be applied to determine which points are actual edge points. Thresholding is often utilized to do so.
 
\end{itemize}

Well known smoothing filters are :

\begin{itemize}

 \item Sobel-Operator
 \item Canny Edge Detector
 \item Laplacian-Filter
 \item Prewitt-Operator
 
 \end{itemize}
 
In this master's thesis, the Sobel Operator was utilized, so it will be described in more detail.

%
\subsection{Sobel Operator}\label{sec:Sobel Operator}

The Sobel Operator, sometimes called the Sobel filter is one of the most used edge detectors in image processing and computer vision. The Sobel Operator uses vertical and horizontal masks. These masks used are odd-numbered square matrices and they are generally 3x3 matrices. Approximations of the derivatives for the horizontal changes and for the vertical changes are calculated by the operator by using two 3x3 kernels and convolving them with the original image. If A is defined as the source, if $G_{x}$ is an image which contains the horizontal derivative approximations at each point, and if $G_{y}$ is an imagine which contains the vertical derivative approximations at each point, then the calculations are:


\[
  G_{x} = 
  \begin{bmatrix}
	+1 & 0 & -1 \\
	+2 & 0 & -2 \\
	+1 & 0 & -1 \\
   \end{bmatrix} * A  \quad
  G_{y} = 
  \begin{bmatrix}
	+1 & +2 & +1 \\
	0 & 0 & 0 \\
	+1 & -2 & -1 \\
  \end{bmatrix} * A
\]

where * here denotes the 2-dimensional signal processing convolution operation.

Since the Sobel kernels can be decomposed as the products of an averaging and a differentiation kernel, they compute the gradient with smoothing. For example, \textbf{$G_{x}$}  can be written as

$  \begin{bmatrix}
	+1 & 0 & -1 \\
	+2 & 0 & -2 \\
	+1 & 0 & -1 \\
   \end{bmatrix} =  \begin{bmatrix}
	1 \\
	2 \\
	1 \\
  \end{bmatrix} \begin{bmatrix}
	+1 & 0 & -1
  \end{bmatrix}
$

The x-coordinate is defined here as increasing in the 'right'-direction, and the y-coordinate is defined as increasing in the 'down'-direction. At each point in the image, the resulting gradient approximations can be combined to give the gradient magnitude, using:

G = $\sqrt{ G_{x}^{2} + G_{y}^{2} }$

Using this information, we can also calculate the gradient's direction:

$\theta = atan(\dfrac{G_{y}}{G_{x}})$

where, for example, $\theta$ is 0 for a vertical edge which is lighter on the right side.\cite{Sobel_Operator}






\begin{figure}[H]
  \centering
  \subfloat[Original Image]{\includegraphics[width=0.4\textwidth]{./Bilder/Sobel_Original.png}\label{fig:f1}}
  \hfill
  \subfloat[Sobel Operator applied Image]{\includegraphics[width=0.4\textwidth]{./Bilder/Sobel_Operator.png}\label{fig:f2}}
  \caption{Sobel Operator\cite{Sobel_Operator}}
\end{figure} 


%
\subsection{Canny Edge Detector}\label{sec:Canny Edge Detector}

Canny edge detector was developed in 1986 and called with the name of its developer John F. Canny. Canny edge detector is also so popular edge detector like Sobel operator. Canny edge detector is a multi-stage algorithm and it can be analyzed in 5 different stages.\cite{Canny_Edge_Detector2}

\begin{enumerate}
\item \textbf{Noise Reduction : } To get so stabile lane detection results, we have to reduce/remove all noise from frames. Lane detection without filtering out the noise can cause false detection. Gaussian filter is used for removing noise in the frames. Gaussian filter blurs images and removes detail and noise. The size of Gaussian filter kernel must be (2k+1)x(2k+1). It is important to choose the size of Gaussien filter because if the size of kernel is larger, detector's sensitivity to noise is lower but on the other hand, with the increase in size of the Gaussian filter kernel, the localization error in the edge detection will also increase slightly. \cite{Canny_Edge_Detector}



\item \textbf{Finding Intensity Gradient of the Image : } Essentially, the Canny algorithm locates edges in image where the grayscale intensity changes most starkly. In order to find these areas, the gradients of the image must be determined. In order to determine the gradients at each pixel in the smoothed image, the Sobel operator is applied. The Sobel operator has already been thoroughly discussed in section \ref{sec:Sobel Operator}.

\item \textbf{Non-maximum Suppression : } Non-maximum suppression is an edge thinning technique which is used as an intermediate step in many computer vision algorithms. The image is scanned along the image gradient direction, and pixels that are not part of the local maxima are set to zero. This way, all image information that is not part of the local maxima is effectively suppressed.

\item \textbf{Double Thresholding : } The edge pixels remaining after applying non-maximum suppression provide a more accurate depiction of real edges in an image. Despite this, there are still some remaining edge pixels resulting from noise and color variation. Therefore, it is necessary to filter out edge pixels with a weak gradient value while preserving edge pixels with a high gradient value. In order to do this, high and low threshold values must be selected. Edge pixels are marked as strong edge pixels when gradient values are higher than the high threshold value. They are marked as weak edge pixels when gradient values are lower than the high threshold value and higher than the low threshold value. They are suppressed when their values are lower than the low threshold value. The two threshold values are determined empirically and are dependent on the content of a given image.


\item \textbf{Hysteresis Thresholding : } Hysteresis Thresholding is the last part of Canny Edge Detector. Until this step, strong edge pixels are extracted from the true edges but there are also some weak edge pixels, some of them are extracted from true edges and some of them are extracted from some noise. So the weak edge pixels which are extracted from true edge, should be strong edge pixels and the weak edge pixels which are extracted from noises must be removed. If there is a weak edge pixel, 8 neighbour pixels of that weak edge pixel is checked and if at least, one pixel of these neighbour pixels is a strong edge pixel then, this weak edge pixels stay as edges in the end picture. 

\end{enumerate}


\begin{figure}[H]
  \centering
  \subfloat[Original Image]{\includegraphics[width=0.4\textwidth]{./Bilder/Sobel_Original.png}\label{fig:f1}}
  \hfill
  \subfloat[Canny Edge Detector applied Image]{\includegraphics[width=0.4\textwidth]{./Bilder/Canny_Edge_Detector.png}\label{fig:f2}}
  \caption{Canny Edge Detector\cite{Canny_Edge_Detector}}
\end{figure} 





%
\section{Hough-Transformation}\label{sec:Hough-Transformation}
%

%
\subsection{Standart Hough-Transformation}\label{sec:Standart Hough - Transformation}
%


%
\subsection{Probabilistic Hough-Transformation}\label{sec:Probabilistic Hough-Transformation}
%




%
\section{K-Nearest Neighbors Algorithm}\label{sec:K-Nearest Neighbors Algorithm}
%
K-Nearest Neighbor(KNN) is an non-parametric lazy learning algorithm. The non-parametric technique means that it doesn't make any assumptions on the underlaying data distribution. In the definition of KNN, the term 'lazy learning algorithm' is used. It means it doesn't use the data training points to do any generalization. In other words, there is no explicit training phase or it is very minimal.  It also means that the training phase is pretty fast. Most of the lazy algorithms - especially KNN - make decisions based on the entire training data set. On the other hand, KNN is one of the top 10 data mining algorithms\cite{k_nearest_neighbors}.

\begin{figure}[H]
 \centering
  \includegraphics[width=1\textwidth]{./Bilder/k-nearest-neighbors.png}\label{Procedures_of_IPM}
  \caption{K-Nearest-Neighbors Algorithm\cite{k_nearest_neighbors_wikipedia}}
\end{figure}

The K-Nearest Neighbors Algorithm has advantages and disadvantages. According to \cite{k_nearest_neighbors_adv_disadv},the main advantages of KNN are simplicity, effectiveness, intuitiveness and competitive classification performance in many domains. On the other hand, KNN can have poor run-time performance when the training set is large. It is very sensitive to irrelevant or redundant features because all features contribute to the similarity and thus to the classification. The computation cost is also quite high because we need to compute distance of each query instance to all training samples. 

%
\section{Curve Fitting}\label{sec:Curve Fitting}

\emph{\color{red}Curve fitting is used to find the 'best fit' line or curve for a series of data points. Curve fitting produces mostly mathematical equations that can be used to find points anywhere along
the curve.\cite{Curve_Fitting}} They are several different types of curve fitting. Some of them are: linear, exponential, polynomial, exponential, power, logarithmic, etc. In this master thesis, polynomial curve fitting was used. Polynomial curve fitting differs from order of the polynomial. Polynomial curve fittings are called different names depending on their orders. First order polynomial curve fittings are called linear regression, second order as quadratic regression, and third order as cubic regression.


\begin{figure}[H]
 \centering
  \includegraphics[width=1\textwidth]{./Bilder/Curve_Fitting_Polynomial.png}\label{Curve_Fitting_Polynomial}
  \caption{Types of Polynomial Curve Fitting\cite{Curve_Fitting_Polynomial}}
\end{figure}


\emph{\color{blue}In this master thesis, curve fitting is used for getting the best mathematical descriptions of lanes. Curve fitting uses as input the Hough points, which appear on the lanes and give as output a mathematical equation. First order polynomial curve fitting is more suitable for defining straight lines and second order polynomial curve fitting is more suitable for defining curves. There are straight and curve lanes so first order polynomial curve fitting wouldn't be enough for our project because of this reason, in this master thesis, second order polynimal curve fitting is used.}

\emph{\color{blue}There are so many different methods for curve fitting. One of the most famous method is the least squares method. In this master thesis also the least squares method is used. Next, it will be described, how to generate a polynomial curve fitting with using the least squares method.}

\emph{\color{blue}A data set can be mostly expressed the relationship between variable with an equation which is mostly representated with a $k^{th}$ order polynomial. The general description of $ k^{th} $ is :}

y = $ a_{k}x^{k} + ... + a_{1}x + a_{0} + \epsilon $ 

\emph{\color{blue}The general polynomial regression model with the error $\epsilon$ provide typically an estimate rather than an implicit value of the dataset for any given value of x. A data set which has N data points, can be described with the maximum order of the polynomial which is k = N - 1 but mostly the lowest possible order of polynomial is choosed.}

\emph{\color{blue} The aim of the least squares' method is to minimise the variance between dataset values and estimated values from the polynomial equation.}

\emph{\color{blue} For determining the coefficients of the polynomial regression model ($ a_{k}, a_{k-1}, ..., a_{1} $) must be solved the following linear equations.}

 $
 \begin{bmatrix}
N & $$\sum_{i=1}^{N} x_{i}$$ & $\dots$ & $$\sum_{i=1}^{N} x_{i}^{k}$$ \\
$$\sum_{i=1}^{N} x_{i}$$ & $$\sum_{i=1}^{N} x_{i}^2$$ & $\dots$ & $$\sum_{i=1}^{N} x_{i}^{k+1}$$ \\
$\vdots$ & $\vdots$ & $\vdots$ & $\vdots$ \\
$$\sum_{i=1}^{N} x_{i}^{k}$$ & $$\sum_{i=1}^{N} x_{i}^{k+1}$$ & $\dots$ & $$\sum_{i=1}^{N} x_{i}^{2k}$$ \\

\end{bmatrix}  \begin{bmatrix}
	 a_{0}  \\
	 a_{1}  \\
	 \vdots  \\
	 a_{k}  \\
  \end{bmatrix} = 
  \begin{bmatrix}
	 $$\sum_{i=1}^{N} y_{i}$$  \\
	 $$\sum_{i=1}^{N} x_{i}y_{i}$$  \\
	 \vdots  \\
	 $$\sum_{i=1}^{N} x_{i}^{k}y_{i}$$  \\
  \end{bmatrix}
$ 


%





%
\chapter{Implementation}\label{cha:Implementation}
%
\section*{3.1.Test Track}\label{sec:Test Track}
\addcontentsline{toc}{section}{3.1.Test Track}

As before mentioned, medium-term goal of this master thesis is attending to the Carola-Cup at Braunschweig University 
so the test truck was prepaid in the Carola-Cup properties by Nicolas Acero Sepulveda, who did his bachelor thesis also
with this model auto. For this test truck, two black PVC floor carpets were used and on these floor carpets, the lanes 
of the truck were made by using white electrical tape. The straight part of the truck was made on one of these PVC 
floor carpet and the curved part of truck was made on second PVC floor carpet. The straigt part of the truck is 
approximately 2 meters long and the curve radius of the curved part of test truck is approximately 1 meter. This curve 
is the tightest curve at Carola-Cup so with this test truck can be tested the worst case situation. In the Carola-Cup 
competition, the truck is much more bigger but for testing this master thesis, we don't need to build bigger test truck.


%
\section*{3.2.Hardware}\label{sec:Hardware}
\addcontentsline{toc}{section}{3.2.Hardware}

%
\subsection*{3.2.1. Model Auto}\label{sec:Model Auto}
\addcontentsline{toc}{subsection}{3.2.1. Model Auto}

During the course of this master thesis, a model automobile was being used which was prepared for the Projectseminar 
Echtzeitsysteme at Technical University of Darmstadt. The chassis, steering mechanism, power train, and engine control 
were derived from the model-building of a Japanese company, Tamiya. The maximum velocity of the model automobile is 
approximately 1 m/s and the minimum steering radius is around 90 cm. 

%

\subsection*{3.2.2. Microcontroller and Main Board}\label{sec:Microcontroller and Main Board}
\addcontentsline{toc}{subsection}{3.2.2. Microcontroller and Main Board}
%

\subsection*{3.2.3. Camera}\label{sec:Camera}
\addcontentsline{toc}{subsection}{3.2.3. Camera}

The camera is one of the main components of lane detection and accordingly, autonomous driving. For this thesis, I had 
to research the most suitable camera because all cameras have different properties.

At the beginning of the Projectseminar Echtzeitsysteme, the Logitech C270 HD Webcam was being used. The resolution of 
the camera is 1280x960 pixels and the Frame per Second (FPS) value is 30 Hertz (Hz) at a 640x480 pixel resolution. 
The field of View (FOV) is just 60 degrees. The problem with this camera is that if there is a curve, the camera 
cannot see all of the lanes, and thus is not very suitable for lane detection. When I started my master thesis, there 
was a Kinect v2 camera on the model car.  The Kinect v2 camera was developed by Microsoft and released in 2013. This 
camera has a depth sensor with a resolution of 512x424 pixels and its FOV is 70x60 degrees. The FPS value is 30 Hz at 
a 512x424 pixel resolution. This camera also has a color camera with resolution of 1920x1080 pixels and a FOV of 
84.1x53.8 degrees. The FPS value is 30 Hz at a 1920x1080 pixel resolution. This camera had two main disadvantages for 
this master thesis. The first disadvantage is the FOV value of camera. This value is better than the value of Logitech 
C270 camera but it is still not enough for curve lane detection. The second main disadvantage is the location of the 
color camera. The color camera of this camera is not in the middle of camera, but rather, on the right. This is a 
disadvantage for us because when there are curves going left as opposed to right, the camera is unable to see the 
left and even perhaps the middle lane of the truck. Thus, this is problematic for lane detection.

Due to these reasons, I had to choose a camera which has a sufficiently high FOV value. After doing research, I decided 
that the Genius Widecam F100 camera is the best choice for this master thesis because this camera has a FOV value of 
120 degrees and it can also be used with the Linux Operating System. The resolution of this camera is 1920x1080 pixels 
and the FOV value is 120 degrees. The FPS is 30 Hz at a 1920x1080 pixel resolution. With this camera, it is possible 
to detect most if not all lanes, including when there are curves. 

\begin{figure}
	\centering
	\hspace*{0cm}   
	\includegraphics[width=150mm,scale=1]{./Bilder/Genius_F100_camera.png}
	\caption{Genius 120-degree Ultra Wide Angle Full HD Conference Webcam(WideCam F100) }
\end{figure}


%
\section*{3.3.Software}\label{sec:Software}
\addcontentsline{toc}{section}{3.3.Software}
%



%
\chapter{Evaulation and Discussion}
\label{cha:Evaulation and Discussion}

\emph{\color{blue}In this chapter, the algorithms for each case will be evaluated, which are used in this master thesis. Also the parameters in the algorithms will be changed and their effects to the computing time and to the reliability of lane detection will be observed. The cases will be also compared with each other and trid to find the most efficient and reliable case. One of the part of this thesis, the average computation times for each algorithms and for the total lane detection process for each cases will be also measured. End of this chapter, the problems which can occur, during the lane detection will be defined.}
%

\section{Average Computing Time}\label{sec:Average Computing Time}



\section{Test Driving}\label{sec:Test Driving}

%
\chapter{Related Works}\label{cha:Related Works}

\emph{\color{blue}Autonomous driving and accordingly lane detection is so active researching area in automotive industry. So in this topic, so many projects were done and about that so many papers were written. This project can be done with so many methods. These methods have some advantages and disadvantages. For example, some of them can detect just the straight lanes, some of them are not rubust in rainy wheather or in different light conditions. There are also some papers, which compare the results of lane detection projects. One of them is from Ammu M Kumar and Philomina Simon from India, which compare some lane detection projects from 2003 until 2014. }\cite{Review_of_Lane_Detection}
 
\emph{\color{blue}Papers show that, one of the most used methods for lane detection is 'Hough Transformation'. But there are also some other methods like 'Haar like features', or Random sample consensus(RANSAC) or Artificial Neural Network(ANN). In this chapter, some different methods will be described and it can give some overview about other implementation methods.} 

\emph{\color{blue}An algorithm for lane detection from Mohamed Aly, which was also used by Nicolas Acero Sepulveda, who did his bachelor's thesis about lane detection at Technical University of Darmstadt in 2016.\cite{Bachelorthesis_Nicolas} In this method, at the beginning with the Inverse Perspective Mapping(IPM) algorithm, the view was changed from camera view to top view. With it, the perspective effect is avoided. For the top viewed frame, Gaussian and threshold filters are used for filter noise. Until now, this method is not so big different than then Method 2 in this master's thesis. In this master's thesis, for detecting the lanes, Probabilistic Hough Transformation was used but in the algorithm from Aly, RRandom sample consensus(RANSAC) algorithm was used for detecting the lanes.}

\emph{\color{blue}In another method\cite{An_Efficient_Lane_Detection} for lane detection, the Haar like features were used to obtain candidate lane points. According to autors, the processing time of this method is approximately 0.12 ms. and the detection rate is 90.16\%. It is one of the fastest algorithm for lane detection. In this method, because of the perspective effect, the lanes appear diagonal so a diagonally directional filter is used, which is called 'steerable filter', and then 'Haar like features'is implemented. After this process, the maximum responses are obtained, which mean, that they are left and right lanes.} 

\emph{\color{blue}An another robust method\cite{A_Fast_and_Robust_Approach}, which was developed at the Technical University of Braunschweig, was tested successfully in the competition for the autonomous cars in 2007, which is called 'DARPA Urban Challenge'. 'DARPA Urban Challenge' is different than Carolo Cup, because the Carolo-Cup is for a 1:10 concept of an automa-
ted vehicle but the DARPA Urban Challenge is for real vehicles. In this project, the IPM technik was used and instead of the RGB color space, the HSV color space was used because this format is much more robust for different light conditions. In this project, like in the Aly's project, the RANSAC algorithm was used. According the paper for this project, just 10 Frame per second can be processed and this project needs to have a modern graphics card. } 


%
\chapter{Conclusion}\label{cha:Conclusion}
%

The aim of this master's thesis is to implement a method, which detect the lanes specifically for indoor scenarios, which is necessary for the Carolo-Cup. On the other hand, the compution time shouldn't be so high and the project has to work stabil in the expected light conditions. 

To reach the aim, so much projects about lane detection were researched and their advantages and disadvantages were compared with each other. In the literatur, there are so many different method for lane detection and all of them have advantages and disadvantages  	dependent on the application scenario and other factors(detecting just straight lanes or straight and curve lanes, power of the CPU, needed FPS value, and so on). In this case, 5 different methods were implemented. In  3 of the 5 methods, Inverse Perspective Mapping algorithm is used which diminish the view effect but conversely, the computing time of the IPM algorithm is so high. 2 of the 5 methods show that, the lane detection algorithm can be also implemented without IPM algorithm. In all methods, preprocessing parts (Edge detection, threshold filter and so on) are similar. In all methods, for the detecting the pixels which are on the lanes, the Hough Transformation was used. End of the Hough Transformation implementation, the pixels which are close to each other have to been found and for this task, two different algorithms were used which are called K-Nearest Neighbors algorithm and Rectangle algorithm. After all pixels on the lanes are grouped for each lanes, the curve fitting algorithm is used for all lanes in all methods.

After the measurement of the computing times for all algorithms is noticed, that the preprocessing part, the IPM algorithm and Probabilistic Hough Transformation are the most computationally expensive algorithms. In some methods, computing time of these algorithms were descreased. In some methods, the IPM algorithm is implemented just for the fitted curves, in the other words, the IPM algorithm in Method 1 and Method 4 is implememented just for the approximatly 0.5\% of the pixels in the frame so it saves so much computing time for the IPM algorithm.  Another case with the computing time is to use two different algorithms for the grouping the Hough points after the Probabilistic Hough Transformation was used. K-Nearest Neighbors Algorithm takes less computing time to compare to Rectangle Method. The last thing, which can be concluded about computing time is the resolution of the frames. If the resolution of the frames is discreased, the number of the pixels which have to be worked on is also discreased so, the processes take less time.


% =================================================================================
% Anhang
% =================================================================================
%\appendix % Damit wird der Anhang begonnen. Die Kapitel werden ab jetzt mit Buchstaben nummeriert

%%
%\clearpage 
\appendix 
%\addcontentsline{toc}{chapter}{Anhang} 
%\addtocontents{toc}{% 
 % \protect\addtokomafont{chapterentry}{Anhang\ } 
%} 
\chapter{Appendix} 
\section{Default Values of Parameters} 


\begin{center}
  \begin{tabular}{ | c | c | c | }
    \hline
    Parameter & Identification				  &  Standard Value   \\ \hline
    $ Y^{*} $ & Height of IPM picture  		  &  -2   \\ \hline
    $ X^{*} $ & Width of IPM picture  		  &  0.8  \\ \hline
    $ f_{x} $ & x-value of local length 	  &  318.503  \\ \hline
    $ f_{y} $ & y-value of local length 	  &  318.266  \\ \hline
    $ c_{x} $ & x-value of optical center	  &  320.129  \\ \hline
    $ c_{y} $ & y-value of optical center	  &  208.651  \\ \hline
    h		  & 0.9 &  0.8  \\ \hline
    $ \alpha $& 0.9 &  0.8  \\ \hline
    $ \beta $ & 0.9 &  0.8  \\ \hline

\label{tab:parameters}
  \end{tabular}
\end{center}

%

%\input{inc/commonmacros_desc.tex}

% =================================================================================


%% =================================================================================
%% Abbildungsverzeichnis
%% =================================================================================
%\cleardoublepage
%\phantomsection					% F�r Aufnahme ins Inhaltsverzeichnis
%\addcontentsline{toc}{chapter}{\listfigurename}	% In Inhaltsverzeichnis von
%												% Dokument und pdf aufnehmen
%\listoffigures
%% =================================================================================
%
%% =================================================================================
%% Tabellenverzeichnis
%% =================================================================================
%\cleardoublepage
%\phantomsection					% F�r Aufnahme ins Inhaltsverzeichnis
%\addcontentsline{toc}{chapter}{\listtablename}	% In Inhaltsverzeichnis von
%												% Dokument und pdf aufnehmen
%\listoftables
%% =================================================================================

% =================================================================================
% Literaturverzeichnis
% =================================================================================
\cleardoublepage
\phantomsection					% F�r Aufnahme ins Inhaltsverzeichnis
\addcontentsline{toc}{chapter}{\bibname}	% In Inhaltsverzeichnis von
											% Dokument und pdf aufnehmen
%\bibliographystyle{gerabbrv}	% Verweise nummeriert in eckigen Klammern, alphabetisch sortiert
\bibliographystyle{gerunsrt}	% Verweise nummeriert in eckigen Klammern, nach Erscheinung sortiert
\bibliography{./bib/literature}	% Literaturverzeichnis einf�gen, mit Angabe der
								% Bibtex-Datei
%



%
%\clearpage 
\appendix 
%\addcontentsline{toc}{chapter}{Anhang} 
%\addtocontents{toc}{% 
 % \protect\addtokomafont{chapterentry}{Anhang\ } 
%} 
\chapter{Appendix} 
\section{Default Values of Parameters} 


\begin{center}
  \begin{tabular}{ | c | c | c | }
    \hline
    Parameter & Identification				  &  Standard Value   \\ \hline
    $ Y^{*} $ & Height of IPM picture  		  &  -2   \\ \hline
    $ X^{*} $ & Width of IPM picture  		  &  0.8  \\ \hline
    $ f_{x} $ & x-value of local length 	  &  318.503  \\ \hline
    $ f_{y} $ & y-value of local length 	  &  318.266  \\ \hline
    $ c_{x} $ & x-value of optical center	  &  320.129  \\ \hline
    $ c_{y} $ & y-value of optical center	  &  208.651  \\ \hline
    h		  & 0.9 &  0.8  \\ \hline
    $ \alpha $& 0.9 &  0.8  \\ \hline
    $ \beta $ & 0.9 &  0.8  \\ \hline

\label{tab:parameters}
  \end{tabular}
\end{center}

%


\end{document}

%
 =================================================================================

