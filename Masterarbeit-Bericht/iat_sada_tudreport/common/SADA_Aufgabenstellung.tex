\section*{Aufgabenstellung}
Im Rahmen des Projektseminars Echtzeitsysteme werden von Studenten Themen aus 
dem Bereich des autonomen Fahrens bearbeitet. Bisher haben sich die
Fahrzeuge dabei frei im Raum bewegt, und eine Orientierung st�tzte sich im 
Wesentlichen auf W�nde, die �ber Ultraschallsensoren und 2D- sowie 3D-Kameras
erkannt werden.

Um realistischere Fahrsituationen zu behandeln, soll in Zukunft auf eine Fahrt
innerhalb von markierten Fahrspuren umgestellt werden. Als mittelfristiges Ziel
ist die Teilnahme studentischer Gruppen am Carolo-Cup zu nennen. Die Spezifikation 
der Fahrspurmarkierungen ist daher den Regeln zum Carolo-Cup zu
entnehmen.

Basisziel dieser Arbeit ist die Implementierung einer geeigneten Methode, die
anhand von Kameradaten die aktuelle Fahrspur und die Nachbarfahrspur erkennt.
Dabei ist der Verlauf der Fahrspur in einer geeigneten mathematischen Beschreibung 
anzugeben, aus der sich die Breite der Fahrspur, deren Kr�mmung und Kr�mmungs�nderung 
�ber den Weg bestimmen l�sst.

Damit die Daten sinnvoll weiterverarbeitet werden k�nnen, ist eine gen�gend
kleine Abtastzeit zu erreichen, und die Totzeit zwischen der Bilderfassung und der
Ausgabe der Ergebnisse darf nicht zu hoch werden. Zudem sollte bei der in einem 
Zeitschritt bestimmten Fahrspur angegeben werden, wie sich diese in Relation
zur der im Schritt davor bestimmten Fahrspur verh�lt, um mit einem ortsfesten
Koordinatensystem rechnen zu k�nnen.

Diese Erkennung soll ausreichend robust sein, dass diese in R�umen sicher funktioniert. 
D. h. es m�ssen die typischerweise zu erwartenden Lichtbedingungen ber�cksichtigt werden.

Eine einfache Fahrzeugf�hrung ist ebenfalls Bestandteil der Arbeit und notwendig, um die 
Fahrspurerkennung validieren zu k�nnen. Diese ist jedoch nicht
Schwerpunkt der Arbeit. Die Fahrzeugf�hrung soll ein fl�ssiges, nicht zu langsames Fahren 
innerhalb der erkannten Fahrspur erm�glichen.

Die Verwendung bestehender L�sungen ist m�glich und wird \texttt bei entsprechenden guten 
Ergebnissen \texttt auch positiv bewertet. Damit soll aber eine entsprechende
Erweiterung wie

\begin{itemize}

\item das Erkennen von Kreuzungen,

\item das Erreichen einer gewissen Robustheit gegen�ber fehlerhaften (d. h. unterbrochenen) Markierungen (Hierbei sollte sich an den Regeln des Carolo-Cups orientiert werden.) und

\item das Erkennen von Schildern

\end{itemize}

einhergehen.

Wenn begr�ndet, k�nnen weitere bzw. andere Kameras zur Verf�gung gestellt werden.

\vspace{0.5cm}
\begin{tabular}{ll}
Beginn: & \SADABegin \\
Ende:   & \SADAAbgabe \\
Seminar:& \SADASeminar
\end{tabular}

\vspace{1cm}

\begin{tabular}{ll}
\rule{7cm}{0.4pt} \hspace{1cm} & \rule{7cm}{0.4pt} \\
\SADAProf & \SADABetreuer\\
 &\SADABetreuerII\\
 &\SADABetreuerIII
\end{tabular}

\vfill
{\renewcommand{\baselinestretch}{1} % f�r diesen Abschnitt einfacher Zeilenabstand
\normalsize % anwenden des Zeilenabstandes
\begin{minipage}{0.8\textwidth}
	Technische Universit�t Darmstadt\\
	\SADAinstitut\\[0.5cm]
%
	Landgraf-Georg-Stra�e 4\\
	64283 Darmstadt\\
	Telefon \SADAtel\\
	\SADAwebsite
\end{minipage}
\begin{minipage}{0.2\textwidth}
\flushright  % rechtsb�ndig
\ \\[2.7cm]
\SADAlogo\;
\end{minipage}}

