\selectlanguage{english}
\maketitle

% Die Farbe der Identit�tsleiste wird auf Grau umgestellt, damit nicht alle Seiten
% farbig gedruckt werden m�ssen
\ifTUDdesign
	\ifOnlyColorFront	% ggf. nachfolgende Balken andere Farbe zuweisen
		\makeatletter 	% ben�tigt, um die @-Befehle auszuf�hren
    \renewcommand{\@TUD@largerulecolor}{\color{tud0b}}% am besten gleiche Farbe wie in der ersten Zeile und die Zahl durch die 0 ersetzen, dann hat das Grau die richtige Intensit�t
    \makeatother
	\fi
\fi


\pagenumbering{roman}	% Bis zum Hauptteil werden r�mische Seitenzahlen verwendet

% =================================================================================
% Spezielle Seiten f�r studentische Arbeiten
% =================================================================================
\cleardoublepage
\section*{Aufgabenstellung}
Im Rahmen des Projektseminars Echtzeitsysteme werden von Studenten The-
men aus dem Bereich des autonomen Fahrens bearbeitet. Bisher haben sich die
Fahrzeuge dabei frei im Raum bewegt, und eine Orientierung st�tzte sich im We-
sentlichen auf W�nde, die �ber Ultraschallsensoren und 2D- sowie 3D-Kameras
erkannt werden.

Um realistischere Fahrsituationen zu behandeln, soll in Zukunft auf eine Fahrt
innerhalb von markierten Fahrspuren umgestellt werden. Als mittelfristiges Ziel
ist die Teilnahme studentischer Gruppen am Carolino-Cup zu nennen. Die Spe-
zifikation der Fahrspurmarkierungen ist daher den Regeln zum Carolino-Cup zu
entnehmen.

Basisziel dieser Arbeit ist die Implementierung einer geeigneten Methode, die
anhand von Kameradaten die aktuelle Fahrspur und die Nachbarfahrspur erkennt.
Dabei ist der Verlauf der Fahrspur in einer geeigneten mathematischen Beschrei-
bung anzugeben, aus der sich die Breite der Fahrspur, deren Kr�mmung und Kr�m-
mungs�nderung �ber den Weg bestimmen l�sst.

Damit die Daten sinnvoll weiterverarbeitet werden k�nnen, ist eine gen�gend
kleine Abtastzeit zu erreichen, und die Totzeit zwischen der Bilderfassung und der
Ausgabe der Ergebnisse darf nicht zu hoch werden. Zudem sollte bei der in ei-
nem Zeitschritt bestimmten Fahrspur angegeben werden, wie sich diese in Relation
zur der im Schritt davor bestimmten Fahrspur verh�lt, um mit einem ortsfesten
Koordinatensystem rechnen zu k�nnen.

Diese Erkennung soll ausreichend robust sein, dass diese in R�umen sicher funk-
tioniert. D. h. es m�ssen die typischerweise zu erwartenden Lichtbedingungen be-
r�cksichtigt werden.

Eine einfache Fahrzeugf�hrung ist ebenfalls Bestandteil der Arbeit und not-
wendig, um die Fahrspurerkennung validieren zu k�nnen. Diese ist jedoch nicht
Schwerpunkt der Arbeit. Die Fahrzeugf�hrung soll ein fl�ssiges, nicht zu langsa-
mes Fahren innerhalb der erkannten Fahrspur erm�glichen.

Die Verwendung bestehender L�sungen ist m�glich und wird \texttt bei entsprechen-
den guten Ergebnissen \texttt auch positiv bewertet. Damit soll aber eine entsprechende
Erweiterung wie

\begin{itemize}

\item das Erkennen von Kreuzungen,

\item das Erreichen einer gewissen Robustheit gegen�ber fehlerhaften (d. h. unterbrochenen) Markierungen (Hierbei sollte sich an den Regeln des Carolino-Cups orientiert werden.) und

\item das Erkennen von Schildern

\end{itemize}

einhergehen.

Wenn begr�ndet, k�nnen weitere bzw. andere Kameras zur Verf�gung gestellt werden.

\vspace{0.5cm}
\begin{tabular}{ll}
Beginn: & \SADABegin \\
Ende:   & \SADAAbgabe \\
Seminar:& \SADASeminar
\end{tabular}

\vspace{1cm}

\begin{tabular}{ll}
\rule{7cm}{0.4pt} \hspace{1cm} & \rule{7cm}{0.4pt} \\
\SADAProf & \SADABetreuer\\
 &\SADABetreuerII\\
 &\SADABetreuerIII
\end{tabular}

\vfill
{\renewcommand{\baselinestretch}{1} % f�r diesen Abschnitt einfacher Zeilenabstand
\normalsize % anwenden des Zeilenabstandes
\begin{minipage}{0.8\textwidth}
	Technische Universit�t Darmstadt\\
	\SADAinstitut\\[0.5cm]
%
	Landgraf-Georg-Stra�e 4\\
	64283 Darmstadt\\
	Telefon \SADAtel\\
	\SADAwebsite
\end{minipage}
\begin{minipage}{0.2\textwidth}
\flushright  % rechtsb�ndig
\ \\[2.7cm]
\SADAlogo\;
\end{minipage}}



\cleardoublepage
\ \\[3cm]	% Diese Zeile erzeugt einen Abstand von 4cm zur ersten Zeile, die nur ein Leerzeichen
			% enth�lt

\ifx\SADAVarianteErklaerung\ETIT
	\section*{Erkl�rung}
	\noindent
	Hiermit versichere ich, dass ich die vorliegende Arbeit ohne Hilfe Dritter und nur mit den angegebenen Quellen und Hilfsmitteln angefertigt habe. Alle Stellen, die aus den Quellen entnommen wurden, sind als solche kenntlich gemacht. Diese Arbeit hat in gleicher oder �hnlicher Form noch keiner Pr�fungsbeh�rde vorgelegen.\vspace*{20mm} \\
	\noindent
	\begin{tabular}{ll}
		\SADAStadt, den \SADAAbgabe	\hspace{1cm}	& \rule{0.4\textwidth}{0.4pt}\\
										            & \SADAAutor
	\end{tabular}
	


\else\ifx\SADAVarianteErklaerung\MBDA
	\section*{Erkl�rungen}
	\noindent
	Hiermit erkl�re ich an Eides statt, dass ich die vorliegende \SADATyp\ mit dem Titel\ \glqq\SADATitel\grqq\ selb\-st�ndig und ohne fremde Hilfe verfasst, andere als die angegebenen Quellen und Hilfsmittel nicht benutzt und die aus anderen	Quellen entnommenen Stellen als solche gekennzeichnet habe.\\
	Diese Arbeit hat in gleicher oder �hnlicher Form noch keiner Pr�fungsbeh�rde vorgelegen.\vspace*{20mm} \\
	\noindent
	\begin{tabular}{ll}
		\SADAStadt, den \SADAAbgabe	\hspace{1cm}	& \rule{0.4\textwidth}{0.4pt}\\
										& \SADAAutor
	\end{tabular}
	
	
	\vspace{40mm}
	\noindent
	Ich bin damit einverstanden, dass die TU Darmstadt das Urheberrecht an meiner \SADATyp\ zu wissenschaftlichen Zwecken nutzen kann.\vspace*{20mm} \\
	\noindent
	\begin{tabular}{ll}
		\SADAStadt, den \SADAAbgabe	\hspace{1cm}	& \rule{0.4\textwidth}{0.4pt}\\
										& \SADAAutor
	\end{tabular}

	{\huge Hier fehlt noch was!}
	
	
\else\ifx\SADAVarianteErklaerung\MBSA
	\section*{Erkl�rungen}
	\noindent
	Hiermit erkl�re ich an Eides statt, dass ich die vorliegende \SADATyp\ mit dem Titel\ \glqq\SADATitel\grqq\ selb\-st�ndig und ohne fremde Hilfe verfasst, andere als die angegebenen Quellen und Hilfsmittel nicht benutzt und die aus anderen	Quellen entnommenen Stellen als solche gekennzeichnet habe.\\
	Diese Arbeit hat in gleicher oder �hnlicher Form noch keiner Pr�fungsbeh�rde vorgelegen.\vspace*{20mm} \\
	\noindent
	\begin{tabular}{ll}
		\SADAStadt, den \SADAAbgabe	\hspace{1cm}	& \rule{0.4\textwidth}{0.4pt}\\
										& \SADAAutor
	\end{tabular}
	
	
	\vspace{40mm}
	\noindent
	Ich bin damit einverstanden, dass die TU Darmstadt das Urheberrecht an meiner \SADATyp\ zu wissenschaftlichen Zwecken nutzen kann.\vspace*{20mm} \\
	\noindent
	\begin{tabular}{ll}
		\SADAStadt, den \SADAAbgabe	\hspace{1cm}	& \rule{0.4\textwidth}{0.4pt}\\
										& \SADAAutor
	\end{tabular}


\else
	{\huge Unbekannte Variante der Erkl�rung!}

\fi\fi\fi






\clearpage
%\section*{Kurzfassung}
Das \LaTeX-Dokument \verb|sada_tudreport| ist eine Vorlage f�r schriftliche Arbeiten (Proseminar-, Projektseminar-, Studien-, Bachelor-, Master- und Diplomarbeiten, \etc) am Institut f�r Automatisierungstechnik der TU Darmstadt. Das
Layout ist an die \emph{Richtlinien zur Anfertigung von Studien- und
Diplomarbeiten}~\cite{Richtlinien} angepasst und durch Modifikation der Klasse \verb|tudreport|
realisiert, so dass in der Arbeit die gewohnten \LaTeX-Befehle benutzt werden
k�nnen. Die vorliegende Anleitung beschreibt die Klasse und gibt grundlegende
Hinweise zum Verfassen wissenschaftlicher Arbeiten. Sie ist au�erdem ein
Beispiel f�r den Aufbau einer Studien-, Bachelor-, Master- bzw. Diplomarbeit.

\textbf{Schl�sselw�rter:} Studienarbeit, Bachelorarbeit, Masterarbeit, Diplomarbeit, Vorlage, \LaTeX-Klasse



\selectlanguage{english}
\section*{Abstract}
The \LaTeX\ document \verb|sada_tudreport| provides a template for student's research
reports and diploma theses (`` Proseminar-, Projektseminar-, Studien-, Bachelor-, Master- und Diplomarbeiten'') at the Institute of
Automatic Control, Technische Universit�t Darmstadt. The layout is adapted to
the \emph{``Richtlinien zur Anfertigung von Studien- und
Diplomarbeiten''}~\cite{Richtlinien} and is implemented by modification of the standard \verb|tudreport|
class, so that common \LaTeX\ commands can be used in the text. This manual
describes the class and dwells on general considerations on how to write
scientific reports. Additionally, it is an example for the structure of a
thesis.

\textbf{Keywords:} Research reports, diploma theses, template, \LaTeX\ class
\selectlanguage{ngerman} 
% =================================================================================

% =================================================================================
% Inhaltsverzeichnis
% =================================================================================
\cleardoublepage	% Auf einer leeren rechten Seite beginnen
\phantomsection		% Diese und die n�chste Zeile dient dazu, f�r das Inhalts-
					% verzeichnis einen Eintrag in das pdf-Inhaltsverzeichnis,
					% aber nicht in das normale Verzeichnis zu erzeugen.
\pdfbookmark[0]{\contentsname}{pdf:toc}	
\tableofcontents	% Inhaltsverzeichnis einf�gen
\clearpage	% Sonst kommt nichts mehr auf die Seite
% =================================================================================


% =================================================================================
% Symbole und Abk�rzungen
% =================================================================================
% Nach dem Inhaltsverzeichnis kommt ein Verzeichnis der h�ufig verwendeten
% Symbole und Abk�rzungen. Dazu kann man das Paket 'nomencl' verwenden, oder man
% erstellt es von Hand.
%\chapter*{Symbole und Abk�rzungen}
\addcontentsline{toc}{chapter}{Symbole und Abk�rzungen} % erzeugt einen Eintrag im Inhaltsverzeichnis
%
\paragraph*{Lateinische Symbole und Formelzeichen}
\begin{tabularx}{\textwidth}{@{}l@{\qquad}X@{\quad}p{18mm}}
	Symbol & Beschreibung & Einheit\\ \midrule
  $I$ & Strom 			& \unit{A}\\
  $R$ & Widerstand 	& \unit{\Omega}\\
  $U$ & Spannung 		& \unit{V}\\
\end{tabularx}
%
\paragraph*{Griechische Symbole und Formelzeichen}
\begin{tabularx}{\textwidth}{@{}l@{\qquad}X@{\quad}p{18mm}}
	Symbol & Beschreibung & Einheit\\ \midrule
  $\mat{\Psi}$ & Datenmatrix\\
  $\sigma$     & Standardabweichung\\
  $\omega$     & Kreisfrequenz 	& \unit{s^{-1}}
\end{tabularx}
%
\paragraph*{Abk�rzungen}
\begin{tabularx}{\textwidth}{@{}l@{\qquad}X}
K�rzel & vollst�ndige Bezeichnung \\ \midrule
  Dgl. & Differentialgleichung\\
  LS   & Kleinste Quadrate (\emph{Least Squares})\\
  PRBS & Pseudo-Rausch-Bin�r-Signal (\emph{Pseudo Random Binary Signal})\\
  ZVF  & Zustandsvariablenfilter
\end{tabularx}
%
\cleardoublepage

% =================================================================================
% Hauptteil
% =================================================================================
\pagenumbering{arabic}	% Hauptteil bekommt arabische Seitenzahlen